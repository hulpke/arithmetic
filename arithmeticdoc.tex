\documentclass[a4paper,12pt]{amsart}
%\usepackage{fullpage}
\usepackage[pass,a4paper]{geometry}
\usepackage{url,amssymb}
\newcommand{\Z}{\mathbb{Z}}
\newcommand{\SL}{\mathrm{SL}}
\newcommand{\PSL}{\mathrm{PSL}}
\newcommand{\Sp}{\mathrm{Sp}}
\newcommand{\SX}{\mathrm{SX}}
\newcommand{\PSp}{\mathrm{PSp}}
\def\mycmd#1{\ \smallskip\\$\blacktriangleright$\ #1\\}

\title{{\sf GAP} functionality for Zariski dense groups}
\author{Alla Detinko}
\address{School of Computer Science, University of St Andrews, UK}
\author{Dane Flannery}
\address{School of Mathematics, Statistics and Applied Mathematics,
National University of Ireland, Galway, Ireland}
\author{Alexander Hulpke}
\address{Department of Mathematics, Colorado State University, Fort Collins,
CO 80523-1874, USA}
\date{Version 1.4, August 2017}

\begin{document}
\maketitle

In this document we describe the functionality of {\sf GAP} \cite{Gap} 
routines for Zariski dense or arithmetic groups that are developed
in~\cite{Arithm,Density, MFO}.

The research underlying the software was supported through the programme
``Research in Pairs'', at the Mathematisches Forschungsinstitut Oberwolfach,
and part of the software was written during a stay in June 2017.
The hospitality we received has been greatly appreciated. 

Our research was also supported by a Marie Sk\l odowska-Curie Individual
Fellowship grant under Horizon 2020 (EU Framework Programme for
Research and Innovation),
and a Simons Foundation Collaboration Grant Nr. 244502. All support
is acknowledged with gratitude.

\section{Background}

We consider $H$ to be an integral matrix group, given by generator matrices
which lie in the special linear group
\[
\SL(n,\Z)=\left\{A\in\Z^{n\times n}\mid \det A=1\right\}
\]
for $n\ge 2$; respectively the symplectic group
\[
\Sp(n,\Z)=\left\{A\in \SL(n,\Z)\mid A^T\cdot J\cdot A=J\right\}
\]
with 
\[
J=\left(\begin{array}{cc}0&1_{n/2}\\
-1_{n/2}&0\\
\end{array}\right) 
\]
for $n\ge 2$ even.
(It will be required to specify which case is to be considered.)  We shall write $\SX(n,\Z)$ from now on,
denoting either choice.

If $m>1$, we can consider the matrices
as being written over $\Z/m\Z$ and obtain a subgroup of $\SX(n,\Z/m\Z)$.
We denote this reduction homomorphism by
$\varphi_m$.

By~\cite{Matthews},
a subgroup $H \leq \SX(n,\Z)$ is Zariski dense in $\SX(n,{\mathbb R})$, if
and only if the
image group $\varphi_p(H)$ is equal to $\SX(n,p)$ for almost all primes
$p$. We denote by $\Pi(H)$ the finite set of primes $p$ for which
$\varphi_p(H)$ is a proper subgroup of $\SX(n,p)$.

Furthermore, $H$ has finite index in $\SX(n,\Z)$ only if $H$ is Zariski
dense. In this case, $H$ is called {\em arithmetic}.
If also $n\ge 3$, there will be a smallest integer $M$, called the 
{\em level} of $H$, such that the kernel of the reduction modulo $M$ homomorphism $\varphi_M$ (as a map on $\SX(n,\Z)$)
lies in $H$.
We denote by $\pi(M)$ the set of prime divisors of $M$.
\medskip

If $H$ is Zariski dense but not of finite index, there will be a unique
minimal arithmetic group $\bar H$ with $H<\bar H\le\SX(n,\Z)$. We call this
group $\bar H$ the {\em arithmetic closure} of $H$. If $H$ is arithmetic, we
simply set $\bar H=H$.

This naturally poses the following algorithmic questions for a subgroup $H
\leq \SX(n,\Z)$, given by generating matrices: 
\begin{itemize}
\item Test whether $H$ is Zariski dense.
\item For a dense subgroup $H$, determine the set $\Pi(H)$ of relevant
primes.
\item If $H$ is arithmetic and $n\ge 3$, determine its level and index in
$\SX(n,\Z)$.
\item If $H$ is dense and $n\ge 3$, determine its arithmetic closure $\bar{H}$.
\end{itemize}
Deterministic algorithms for these questions are described in
\cite{Arithm,Density, MFO}; this documents describes implementations of
these algorithms.
\medskip

%The main functionality of the routines provided is to determine $\Pi(H)$ as
%well as $M$.
%\smallskip

%[\textbf{AD}: this is my text; AH text is commented out below]
A fundamental result of~\cite{Density} is that $\pi(M) = \Pi(H)$ with possible exceptions for $n\le 4$.
In that case we will have that $\pi(M)=\Pi(H)\cup\{2\}$ and $\varphi_4(H)$ is a proper
subgroup of $\SX(n,\Z/4\Z)$.

%A fundamental result of~\cite{Density} is that $\pi(M)\subset\Pi(H)\cup\{2\}$ with inequality only possible for $n\le 4$.
%In the case of inequality we will have that $\varphi_4(H)$ is a proper subgroup of $\SX(n,\Z/4\Z)$.
\bigskip

The actual code consists of a file that can be read in by {\sf GAP}. It is
available at the web address
\url{http://www.math.colostate.edu/~hulpke/arithmetic.g} 

It requires {\sf
GAP} in release at least 4.8; the
{\tt matgrp} package, as well as the packages ({\tt recog}, {\tt genss},
{\tt orb}, {\tt io}) on which it relies.

Place the file in a directory readable by {\sf GAP} and read it as
\begin{verbatim}
gap> Read("arithmetic.g");
\end{verbatim}
You should get a printout that the routines were loaded, respectively an
error message that a required package could not be loaded.
\bigskip

In the description of the functionality below, the default input to most of
the routines will be a group $H$, generated by a finite number of integral
matrices. Functions also take an optional argument {\em kind}, which
currently may take the values {\tt SL} or {\tt 1} (either is indicating that the
group is to be considered as a subgroup of $\SL(n,\Z)$ ), respectively {\tt
SP} or {\tt 2} indicating it as a subgroup of $\Sp$.
If no kind is indicated, it will be assumed that the kind is $\SL$.

The routines perform minimal validity checks for the input. They will not
check for membership of group elements or correctness of the specified kind.
If the kind is $\Sp(n,\Z)$, the form 
%[\textbf{AD:} should we denote identity as $1_n$ is it in papers?]
\[
\left(\begin{array}{cc}0&1_{n/2}\\
-1_{n/2}&0\\
\end{array}\right) 
\]
is assumed. If the matrices preserve only a different form, incorrect
results might arise. Similarly, if the group given is not generated by
integer matrices, the behavior of the routines is undefined.

\section{Basic Routines}

This section describes routines that can be used to generate $\SL(n,\Z)$ or
$\Sp(n,\Z)$ in matrix form, or as a finitely presented group, as well as for
computing with congruence images of matrix groups.

\mycmd{{\tt SLNZFP(n)}}
constructs an isomorphism from a finitely presented version of $\SL(n,\Z)$
to the natural matrix version. Note that factorization of matrices into
generators is not yet possible.
\begin{verbatim}
gap> hom:=SLNZFP(3);
[ t12, t13, t21, t23, t31, t32 ] ->
[ [ [ 1, 1, 0 ], [ 0, 1, 0 ], [ 0, 0, 1 ] ],
  [ [ 1, 0, 1 ], [ 0, 1, 0 ], [ 0, 0, 1 ] ],
  [ [ 1, 0, 0 ], [ 1, 1, 0 ], [ 0, 0, 1 ] ],
  [ [ 1, 0, 0 ], [ 0, 1, 1 ], [ 0, 0, 1 ] ],
  [ [ 1, 0, 0 ], [ 0, 1, 0 ], [ 1, 0, 1 ] ],
  [ [ 1, 0, 0 ], [ 0, 1, 0 ], [ 0, 1, 1 ] ] ]
\end{verbatim}

\mycmd{{\tt HNFWord(hom,mat)}}
For a homomorphism created through {\tt SLNZFP} and a matrix, this routine
returns a word in the finitely presented group that maps to the desired
image matrix. (This will only work for $\SL$.)
\begin{verbatim}
gap> HNFWord(hom,[ [ 26, -3, 9 ], [ 0, -1, 6 ], [ -3, 0, 1 ] ]);
t23^-1*t12^-9*(t21^-1*t12^2)^2*t21^3*t31^-3*t32^-1*t23^-34
  *(t32^-1*t23^2)^2*t32^-1*t23^-5*t12^12*t13^-72
gap> Image(hom,last);
[ [ 26, -3, 9 ], [ 0, -1, 6 ], [ -3, 0, 1 ] ]
\end{verbatim}

\mycmd{{\tt SPNZFP(n)}}
constructs an isomorphism from a finitely presented version of $\Sp(n,\Z)$
to the natural matrix version. Note that factorization of matrices into
generators is not yet possible.
\begin{verbatim}
gap> hom:=SPNZFP(4);
[ T12, T21, U12, U21, V12, V21 ] ->
[ [ [ 1, 1, 0, 0 ], [ 0, 1, 0, 0 ], [ 0, 0, 1, 0 ], [ 0, 0, -1, 1 ] ],
  [ [ 1, 0, 0, 0 ], [ 1, 1, 0, 0 ], [ 0, 0, 1, -1 ], [ 0, 0, 0, 1 ] ],
  [ [ 1, 0, 0, 1 ], [ 0, 1, 1, 0 ], [ 0, 0, 1, 0 ], [ 0, 0, 0, 1 ] ],
  [ [ 1, 0, 0, 1 ], [ 0, 1, 1, 0 ], [ 0, 0, 1, 0 ], [ 0, 0, 0, 1 ] ],
  [ [ 1, 0, 0, 0 ], [ 0, 1, 0, 0 ], [ 0, 1, 1, 0 ], [ 1, 0, 0, 1 ] ],
  [ [ 1, 0, 0, 0 ], [ 0, 1, 0, 0 ], [ 0, 1, 1, 0 ], [ 1, 0, 0, 1 ] ] ]
\end{verbatim}
%
%[\textbf{AD}: In MFO preprint we call it MatrixGroupModulo; I suggest to change name in the preprint]

\mycmd{{\tt ModularImageMatrixGroup(H,m)}}
returns the matrix group over $\Z/m\Z$ obtained by reducing the coefficients
of the matrices in the group $H$ modulo $m$. 
If $H$ is already defined over a residue
class ring $\Z/k\Z$ and $m\mid k$ this is the obvious reduction; if
$m\not\,\mid k$ the result is undefined.
\begin{verbatim}
gap> h:=Group([ [ [ 26, -3, 9 ], [ 0, -1, 6 ], [ -3, 0, 1 ] ],
>   [ [ -1, 0, 0 ], [ -9, 1, -3 ], [ 3, 0, -1 ] ],
>   [ [ 0, 0, 1 ], [ 1, 0, 9 ], [ 0, 1, 0 ] ] ]);
<matrix group with 3 generators>
gap> a:=ModularImageMatrixGroup(h,9*17);
<matrix group with 3 generators>
gap> a.1;
[ [ ZmodnZObj( 26, 153 ), ZmodnZObj( 150, 153 ), ZmodnZObj( 9, 153 ) ],
  [ ZmodnZObj( 0, 153 ), ZmodnZObj( 152, 153 ), ZmodnZObj( 6, 153 ) ],
    [ ZmodnZObj( 150, 153 ), ZmodnZObj( 0, 153 ), ZmodnZObj( 1, 153 ) ] ]
gap> Display(a.1);
matrix over Integers mod 153:
[ [   26,  150,    9 ],
  [    0,  152,    6 ],
gap> b:=ModularImageMatrixGroup(a,3);
<matrix group with 3 generators>
gap> b.1;
[ [ Z(3), 0*Z(3), 0*Z(3) ], [ 0*Z(3), Z(3), 0*Z(3) ],
  [ 0*Z(3), 0*Z(3), Z(3)^0 ] ]
  [  150,    0,    1 ] ]
gap> Display(b.1);
 2 . .
 . 2 .
 . . 1
\end{verbatim}

The general functionality for matrix groups over residue class rings is
still limited to some extent. When working further with such a group, the first
call should be to the (somewhat technical) function {\tt
FittingFreeLiftSetup}. This will establish, for example, the order of the
group, and set up basic data structures for other calculations.

\begin{verbatim}
gap> FittingFreeLiftSetup(a);;
gap> Size(a);
2251866396672
\end{verbatim}

It is also possible to convert the group to an (isomorphic) permutation
representation, using {\tt IsomorphismPermGroup}, but in general these groups will be of inconveniently large
degree.

\section{Density Testing}

This section describes a variety of routines that can be used to test
Zariski density of a matrix group.

\mycmd{{\tt IsTransvection(t)}}
tests whether $t$ is a transvection; that is, the rank of $t-1_n$ is $1$ and
$(t-1_n)^2=0$.
\begin{verbatim}
gap> t:=[ [ 1, 0, 0 ], [ -9, 1, 0 ], [ -6, 0, 1 ] ];;
gap> IsTransvection(t);
true
\end{verbatim}

In all examples below we will assume that
\begin{verbatim}
gap> h:=Group([ [ [ 26, -3, 9 ], [ 0, -1, 6 ], [ -3, 0, 1 ] ],
>   [ [ -1, 0, 0 ], [ -9, 1, -3 ], [ 3, 0, -1 ] ],
>   [ [ 0, 0, 1 ], [ 1, 0, 9 ], [ 0, 1, 0 ] ] ]);
\end{verbatim}
This is a dense subgroup of $\SL(3,\Z)$ and contains the transvection $t$
from the previous example. If we take instead the subgroup generated by the
first generator of $h$ together with $t$, we obtain a subgroup that is no
longer dense.

\mycmd{{\tt IsDenseDFH(H[,kind],t)}}
implements the density test of~\cite{Density}.
%[\textbf{AD}: I added reference to the preprint as the case $\SL(2n, \Z)$ with transvection is implemented now].
Here $t$ must be a transvection in $H$, given as a matrix. (Membership of
$t$ in $H$ is assumed and not tested.)
\begin{verbatim}
gap> IsDenseDFH(h,SL,t);
true
gap> IsDenseDFH(Group(h.1,t),SL,t);
false
\end{verbatim}

\mycmd{{\tt IsDenseIR1(H[,kind])}}
implements the Monte-Carlo algorithm \cite[Algorithm~1]{Rivin1} to test
density. It returns {\tt true} if the group is dense, and {\tt false} if it
is not dense or the test failed.

This routine is by far the fastest of the density tests; however, if it
returns {\tt false}, this might indicate either that the group is not dense,
or that a random search failed to find a suitable element.
\begin{verbatim}
gap> IsDenseIR1(h,SL);
true
gap> IsDenseIR1(Group(h.1,t));
false
\end{verbatim}

\mycmd{{\tt IsDenseIR2(H[,kind])}}
implements the deterministic algorithm \cite[p.23]{Rivin1} to test
density by verifying absolute irreducibility of the adjoint representation. It
is the slowest of the density test routines.
Due to the current implementation, it requires $H$ to consist of
integral matrices (though the algorithm itself would also work over the
rationals or number fields).
\begin{verbatim}
gap> IsDenseIR2(h,SL);
true
gap> IsDenseIR2(Group(h.1,t));
false
\end{verbatim}

\section{Primes and Level}

Next, we describe functions that determine the set of primes $\pi(M)$ and
$\Pi(H)$ associated to $H$, as well as the level $M$ of the arithmetic
closure of $H$. All functions in this section assume that $H$ is dense and
might not terminate, or produce meaningless results, if it is not.
%[\textbf{AD}: to add `Here $H$ is dense' or so ?]

\mycmd{{\tt PrimesNonSurjective(H[,kind])}}
takes an integral matrix group $H$ and
returns the set $\pi(M)$ of primes that divide the level $M$ of the arithmetic
closure $\bar H$ of $H$.

At the moment this function is only implemented for $\SL$ in prime degree;
future releases will cover further cases.

(We continue with the same group $H$ in the examples.)
\begin{verbatim}
gap> PrimesNonSurjective(h,SL);
[ 3, 73 ]
\end{verbatim}

We can translate between $\Pi(H)$ and $\pi(M)$ (which is only required in
dimension $\le 4$) by considering the images
modulo $2$ and modulo $4$. This is illustrated in the following example:

\begin{verbatim}
gap> gp:=Group([[[0,0,1],[1,0,0],[0,1,0]],[[1,2,4],[0,-1,-1],[0,1,0]]]);;
gap> PrimesNonSurjective(gp);
[ 2, 3, 5, 19 ]
gap> gp2:=ModularImageMatrixGroup(gp,2);
Group([ <an immutable 3x3 matrix over GF2>,
  <an immutable 3x3 matrix over GF2> ])
gap> FittingFreeLiftSetup(gp2);;Size(gp2);
168
gap> gp4:=ModularImageMatrixGroup(gp,4);;
gap> FittingFreeLiftSetup(gp4);;Size(gp4);
168
\end{verbatim}

We note that the image $\varphi_2(H)$ modulo $2$ has full order
$|\SL(3,2)|$; thus $2\not\in\Pi(H)$.
On the other hand, the image $\varphi_4(H)$ modulo $4$ is a proper
subgroup of $\SL(3,\Z/4\Z)$, since $|\varphi_4(H)|<|\SL(3,\Z/4\Z)|$.
Thus $2\in\pi(M)$.

%[\textbf{AD}: in [Density] it is PrimeForDense]

\mycmd{{\tt PrimesForDense(H,t[,kind])}}
takes an integral matrix group $H$ and returns $\Pi(H)$.
The element
{\tt t} must be a transvection in $H$.
\begin{verbatim}
gap> PrimesForDense(h,t);
[ 3, 73 ]
\end{verbatim}


\mycmd{{\tt MaxPCSPrimes(H,primes,[,kind])}}
takes an integral matrix group $H$ of dimension $\ge 3$ and the set $\pi(M)$ of primes dividing
the level $M$ of $\bar H$ and returns a list of length 2, containing the
level and index of (the arithmetic closure of) $H$.

\begin{verbatim}
gap> MaxPCSPrimes(h,[3,73]);
[ 1971, 33180341688 ]
\end{verbatim}

If only a subset of $\pi(M)$ is given, only these primes are considered,
leading to a divisor of level and index.

\begin{verbatim}
gap> MaxPCSPrimes(gp,[2,3,5,19]);
[ 5700, 242646091084800000 ]
gap> MaxPCSPrimes(gp,[3,5,19]);
[ 1425, 947836293300000 ]
\end{verbatim}

If the option {\tt quotient} is given, the function returns a list of length
three with the extra third entry being the congruence quotient modulo the level
$M$.

\begin{verbatim}
gap> q:=MaxPCSPrimes(h,[3,73]:quotient);
[ 1971, 33180341688, <matrix group of size 5874712004650752 with
    3 generators> ]
gap> Size(SL(3,Integers mod 1971))/Size(q[3]);
33180341688
\end{verbatim}

\section{Further Examples}

\mycmd{{\tt BetaT(t)}}
returns the group $\beta_T(\Gamma)$ of \cite{LongReidI}.
\begin{verbatim}
gap> h:=BetaT(1);
gap> PrimesNonSurjective(h);
[5]
gap> MaxPCSPrimes(h,[5],SL);
[ 5, 31 ]
\end{verbatim}
As the index is small we can verify arithmeticity by coset enumeration:

\begin{verbatim}
gap> hom:=SLNZFP(3);;
gap> w:=List(GeneratorsOfGroup(h),x->HNFWord(hom,x));
[ t13^-1*t31*(t12*t21^-1*t12)^2*t12*t32*t23^-2,
  t21*t31^-1*(t21^-1*t12^2)^2*t23^-1*(t32^-1*t23^2)^2,
  t12^-1*t21*t23^-1*t32*t23^-1*t13^2 ]
gap> u:=Subgroup(Source(hom),w);
Group([ t13^-1*t31*(t12*t21^-1*t12)^2*t12*t32*t23^-2,
  t21*t31^-1*(t21^-1*t12^2)^2*t23^-1*(t32^-1*t23^2)^2,
  t12^-1*t21*t23^-1*t32*t23^-1*t13^2 ])
gap> Index(Source(hom),u);
31
\end{verbatim}

Such a calculation will be increasingly difficult as the index grows; for
example determining the index $3670016$ of $\beta_{-2}(\Gamma)$ takes a
couple of minutes and any attempt for $\beta_{7}(\Gamma)$ will be hopeless
because of the large index.

\mycmd{{\tt RhoK(k)}}
returns the group $\rho_k(\Gamma)$ of \cite{LongReidI}.

\mycmd{{\tt HofmannStraatenExample(d,k)}}
returns the group $G(d,k)$ of \cite{HofStrat}.
\begin{verbatim}
gap> h:=HofmannStraatenExample(12,7);;IsTransvection(h.2);
true
gap> PrimesForDense(h,h.2,SP);
[ 2, 3 ]
gap> MaxPCSPrimes(h,[2,3],SP);
[ 288, 2388787200 ]
\end{verbatim}

\bibliographystyle{amsplain}

\begin{thebibliography}{10}

\bibitem{Arithm}
A.~S. Detinko, D.~L. Flannery, and A.~ Hulpke, \textit{Algorithms for
arithmetic groups with the congruence subgroup property}, J. Algebra \textbf{421} (2015), 234--259.

\bibitem{Density}
A.~S. Detinko, D.~L. Flannery, and A.~ Hulpke, 
\textit{Zariski density and computing in arithmetic groups}, Math. Comp.,
\url{https://doi.org/10.1090/mcom/3236}.

\bibitem{MFO}
A.~S. Detinko, D.~L. Flannery, and A.~ Hulpke, \textit{Experimenting with
Zariski dense subgroups}, %Oberwolfach preprint,
in preparation.

\bibitem{Gap}
The GAP~Group, {GAP} -- {G}roups, {A}lgorithms, and {P}rogramming,
\noindent
\url{http://www.gap-system.org}

\bibitem{HofStrat}
J.~Hofmann and D.~van Straten, \textit{Some monodromy groups of finite
index in $\Sp_4(\Z)$}, \url{http://arxiv.org/abs/1312.3063v1}.

\bibitem{LongReidI}
D.~D.~Long, A.~W.~Reid, \textit{Small subgroups of $\SL(3, \mathbb{Z})$}, 
Exper. Math.  \textbf{20} (2011),
no.~4, 412--425.

\bibitem{Matthews}
C.~Matthews, L.~N.~Vaserstein, and B.~Weisfeiler, \textit{Congruence
properties of Zariski-dense subgroups. I}, Proc. London Math. Soc. (3)
\textbf{48} (1984), no.~3, 514--532.


\bibitem{Rivin1}
I.~Rivin, \textit{Large Galois groups with applications to Zariski
density}, \url{http://arxiv.org/abs/1312.3009v4}

\end{thebibliography}

\end{document}

\mycmd{{\tt ASOrbit}}
\mycmd{{\tt ASStabilizerGens}}

